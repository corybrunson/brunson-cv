\documentclass[10pt,a4paper]{article}

\usepackage{fullpage}
\usepackage{enumitem}
\usepackage{ulem}
\usepackage{tabularx}
\usepackage{hyperref}

\usepackage[T1]{fontenc}
\usepackage{libertine}
\renewcommand*\familydefault{\sfdefault}  %% Only if the base font of the document is to be sans serif

\begin{document}
\frenchspacing
\pagestyle{empty}
\setlength{\parindent}{0cm}

\centerline{\sc Curriculum Vitae}
\vspace{.25cm}
\centerline{\large\bfseries Cory Brunson}

\vspace{.75cm}
Laboratory for Systems Medicine \hfill Office:\ \ (352) 273-8738 \\
University of Florida \hfill Email: {\tt\small jason.brunson@medicine.ufl.edu} \\
PO Box 100225  \hfill LinkedIn: {\tt\small\nolinkurl{jasoncorybrunson}} \\
Gainesville, FL\ \ 32610 \hfill GitHub: {\tt\small\nolinkurl{corybrunson}}
%I am a citizen and permanent resident of the United States.

\vspace{.25cm}
{\sc Research Appointments}
\begin{itemize}[label=$\circ$,nolistsep]
\item
Assistant Professor, Laboratory for Systems Medicine, 2020--
\item
Postdoctoral Fellow, Skeletal, Craniofacial \& Oral Biology Training Program, UConn Health, 2017--2020
\item
Postdoctoral Fellow, Center for Quantitative Medicine, UConn Health, 2014--2017
\item
Research Assistant, Virginia Bioinformatics Institute, Virginia Tech, 2010--2013
\item
Graduate studies and Research Assistant, Virginia Tech, 2005--2013
\end{itemize}

\vspace{.25cm}
{\sc Education}
\begin{itemize}[label=$\circ$,nolistsep]
\item
PhD, Mathematics, Virginia Tech, 2013.
{\itshape Matrix Schubert varieties for the affine Grassmannian.}
Advisor: Mark Shimozono.
\item
MS, Mathematics, Virginia Tech, 2005.
{\itshape On projective planes \& rational identities.}
Advisor: Dan Farkas.
\item
BS, Mathematics; BS, Statistics; Virginia Tech, 2004.
\end{itemize}

%\vspace{.25cm}
%\centerline{
%algebraic geometry $\bullet$ network science $\bullet$ biomedical informatics $\bullet$ topological data analysis
%}
%{\sc Research Interests} \\
%I am developing a research program that draws upon mature foundations in algebraic geometry and topology to address needs arising in the analysis of biomedical data of increasingly high volume, dimension, and heterogeneity.
%To maintain rigor and flexibility both mathematical and scientific, the program exerts broad theoretical and methodological experience, entails sustained collaborations with administrative, clinical, biomedical, and statistical experts, and invites involvement by aspiring and early-career researchers.

%\pagebreak

\vspace{.25cm}
\newcounter{paper}
{\sc Journal Articles}
\begin{enumerate}[label={[\arabic*]},labelindent=1cm,nolistsep]
%\item
%X Zhang, T Hashemian, {\bfseries JC Brunson} (2022) Optimization of localized machine learning on clinical classification tasks. Manuscript under review.
%\item
%{\bfseries JC Brunson}, M Terasaki (2022) Spatial geometry and topology of glomerular capillary networks. Manuscript under review.
%\item
%M Bertolini, {\bfseries JC Brunson}, J Gabriel (2022) Title. Manuscript under review.
%\item
%E Nitch-Griffin, A Peterson, Y Skaf, {\bfseries JC Brunson} (2022) External validity of personalized mortality Prediction. Manuscript under review.
%\item
%Y Skaf, {\bfseries JC Brunson} (2022) Fixed and adaptive landmark sets for finite metric spaces. Manuscript under review.
%\item
%{\bfseries JC Brunson}, J Jaber, S Nandavaram, DC Patel, D Gomez Manjarres (2022) Racial-ethnic disparities in post-transplant outcomes of lung recipients. Manuscript under review.
%\item
%AD Guastello, {\bf JC Brunson}, N Sambuco, LP Dale, NA Tracy, BR Allen, CA Mathews (2022) Predictors of Professional Burnout and Fulfillment in a Longitudinal Analysis on Nurses and Healthcare Workers in the COVID-19 Pandemic. {\itshape J Clin Nurs} ?(?), ?--?.
\item
AD Guastello, {\bfseries JC Brunson}, N Sambuco, LP Dale, NA Tracy, BR Allen, CA Mathews (2022) Predictors of Professional Burnout and Fulfillment in a Longitudinal Analysis on Nurses and Healthcare Workers in the COVID-19 Pandemic. Manuscript under review.
\item
M Terasaki, {\bfseries JC Brunson}, J Sardi (2020) Analysis of the three dimensional structure of the kidney glomerulus capillary network. {\itshape Sci Rep} 10, 20334.
\item
{\bfseries JC Brunson} (2020) ggalluvial: Layered grammar for alluvial plots. {\itshape J Open Source Software} 5(49), 2017.
\item
{\bfseries JC Brunson}, TP Agresta, RC Laubenbacher (2020) Sensitivity of comorbidity network analysis. {\itshape JAMIA Open} 3(1): 94--103.
\item
{\bfseries JC Brunson}, RC Laubenbacher (2018) Applications of network analysis to routinely collected healthcare data: a systematic review. {\itshape J Am Med Inform Assoc} 25(2): 210--221.
\item
{\bfseries JC Brunson}, X Wang, RC. Laubenbacher (2017) Effects of research complexity and competition on the incidence and growth of coauthorship in biomedicine. {\itshape PLOS ONE} 12(3): e0173444.
\item
{\bfseries JC Brunson} (2015) Triadic analysis of affiliation networks. {\itshape Netw Sci} 3(4): 480--508.
\item
{\bfseries JC Brunson}, S Fassino, A McInnes, M Narayan, B Richardson, C Franck, P Ion, R Laubenbacher (2014) Evolutionary events in a mathematics research collaboration network. {\itshape Scientometrics} 99: 973--998.
\item
E Brown, {\bfseries JC Brunson} (2008) Fibonacci's forgotten number. {\itshape College Math J} 39(2): 112--120.
\item
B.A. Reid, U.C. T\"{a}uber, {\bfseries JC Brunson} (2003) Reaction-controlled diffusion: Monte Carlo simulations. {\itshape Phys Rev E} 68: 1--19.
\setcounter{paper}{\value{enumi}}
\end{enumerate}
%
\vspace{.25cm}
{\sc Software}
\begin{enumerate}[label={[\arabic*]},labelindent=1cm,nolistsep]
\setcounter{enumi}{\value{paper}}
%\item
%I Sokolov, {\bfseries JC Brunson} (2022--) {\sffamily mfrpy}: Python module for computing the minimal functional routes of a graph. {\tt\small\nolinkurl{https://github.com/rtmod/mfrpy}}
%\item
%L Sordo Vieira, {\bfseries JC Brunson}, I Sokolov (2022--) {\sffamily pmodpy}: Graph modulus in Python. {\tt\small\nolinkurl{https://github.com/rtmod/pmodpy}}
%\item
%{\bfseries JC Brunson}, L Sordo Vieira (2022--) {\sffamily rmfr}: Calculate Minimal Functional Routes through Expanded Graphs. {\tt\small\nolinkurl{https://github.com/rtmod/rmfr}}
%\item
%{\bfseries JC Brunson} (2020) {\sffamily cerms}: Comparative Effectiveness Research using MarketScan tables. {\tt\small\nolinkurl{https://bitbucket.org/corybrunson/cerms}}
\item
{\bfseries JC Brunson} (2022--) {\sffamily interplex}: Coercion Methods for Simplicial Complex Data Structures, version 0.1.0. {\tt\small\nolinkurl{https://github.com/corybrunson/interplex}}
\item
{\bfseries JC Brunson} (2021--) {\sffamily individuate}: 'tidymodels' Extension for Individualized Models, version 0.0.0.999. {\tt\small\nolinkurl{https://github.com/corybrunson/imtidy}}
\item
M Piekenbrock, {\bfseries JC Brunson}, H Hinnant (2020--) {\sffamily simplextree}: Provides Tools for Working with General Simplicial Complexes, version 1.0.1. {\tt\small\nolinkurl{https://cran.r-project.org/package=simplextree}}
\item
{\bfseries JC Brunson}, B Demkowicz, S Choudhary (2020--) {\sffamily tdaunif}: Uniform manifold samplers for topological data analysis, version 0.1.0. {\tt\small\nolinkurl{https://cran.r-project.org/package=tdaunif}}
\item
{\bfseries JC Brunson}, E Paul (2020--) {\sffamily ordr}: A `tidyverse' extension for ordinations and biplots, version 0.1. {\tt\small\nolinkurl{https://github.com/corybrunson/ordr}}
\item
R Wadhwa, M Piekenbrock, {\bfseries JC Brunson}, X Zhang, J Scott (2019--) {\sffamily ripserr}: Calculate Persistent Homology with Ripser-Based Engines, version 0.2.0. {\tt\small\nolinkurl{https://github.com/rrrlw/ripserr/}}
\item
{\bfseries JC Brunson}, R Wadhwa, J Scott (2018--) {\sffamily ggtda}: ggplot2-Compatible Visualization of Persistent Homology, version 0.1.0. {\tt\small\nolinkurl{https://github.com/rrrlw/ggtda}}
\item
{\bfseries JC Brunson}, QD Read (2015--) {\sffamily ggalluvial}: Alluvial Plots in `ggplot2', version 0.11.1. {\tt\small\nolinkurl{https://cran.r-project.org/package=ggalluvial}}
\item
{\bfseries JC Brunson} (2014) {\sffamily bitriad}: Triadic Analysis of Affiliation Networks, version 0.3. {\tt\small\nolinkurl{https://github.com/corybrunson/bitriad}}
\setcounter{paper}{\value{enumi}}
\end{enumerate}
%
\vspace{.25cm}
{\sc Book Chapters}
\begin{enumerate}[label={[\arabic*]},labelindent=1cm,nolistsep]
\item
{\bfseries C Brunson} (2015) Mythology and Moorings: Science Surveys in Cultural Context. In MO Stephenson, L Kirakosyan (Ed.) {\itshape RE: Reflections and Explorations: Essays on politics, public policy, and governance}. Virginia Tech Institute for Policy and Governance.
\setcounter{paper}{\value{enumi}}
\end{enumerate}
%
\vspace{.25cm}
{\sc Reports}
\begin{enumerate}[label={[\arabic*]},labelindent=1cm,nolistsep]
\item
{\bfseries JC Brunson} (2015) Analysis of increased compound drug prescriptions in Connecticut 2014--2015. Report to the Office of the State Comptroller of Connecticut.
\setcounter{paper}{\value{enumi}}
\end{enumerate}

%\pagebreak

\vspace{.25cm}
{\sc Funding}
\begin{itemize}[label=$\circ$,nolistsep]
%\item
%Foundation for Sarcoidosis Research Pilot Grant, Title, 2022~Jun~20
\item
CTSI Precision Health Initiative Pilot Grant Program, Efficient Modeling of Individualized COVID-19 Mortality Risk, 2022~Mar~11 (pending IRB and NIH approval)
\item
{\bfseries Submitted}. K25 Mentored Quantitative Research Development Award (PA-20-199), Individualized computational modeling to determine outcomes after lung transplantation, 2022~Feb~12
\item
{\bfseries Not funded} (impact score 56). K25 Mentored Quantitative Research Development Award (PA-20-199), Individualized Modeling of COVID-19 Outcomes using Electronic Health Record Data, 2021~Feb~12
\item
{\bfseries Not funded} (impact score 42). F32 Postdoctoral Individual National Research Service Award (PA-18-670), Topological data analysis for patient stratification and outcomes
research, 2018~Apr~8
\item
Skeletal, Craniofacial and Oral Biology Training Grant, NIDCR 5T90DE021989-07 (M Mina), Structural inference and temporal modeling of clinical co-occurrence networks, 2017~Jul~1--2020~Jul~31
\item
Health Center Research Advisory Council (HCRAC) Travel Award \#111854, 2018~Jul~10--13
\item
``Developing a User-Friendly R Package Providing Standardized Coding and Analytic Methods for Comparative Effectiveness Research Using Administrative Healthcare Claims Data'' (C Coleman), 2018~Feb~1--Jul~31
\item
Co--Principal Investigator, ACSB 2015: A Conference on Algebraic and Combinatorial Approaches in Systems Biology, NSF DMS \#1503562 (MP Vera--Licona), 2015~May~22--24
\end{itemize}

\vspace{.25cm}
{\sc Teaching and Mentoring}
\begin{itemize}[label=$\circ$,nolistsep]
\item
Mentor (University of Florida) \\
Signaling pathway analysis via the graph modulus, 2021--: BS~(1) \\
Geometric topology of glomerular capillaries, 2021--: pre-MD~(1), BS~(1) \\
Racial--ethnic disparities in post-transplant outcomes, 2021--: pre-MD~(2) \\
Predictive models of response to treatment for hoarding disorder, 2021--: MS~(1) \\
Similarity-based individualized risk factor analysis, 2020--: pre-MD~(1), MS~(1) \\
Clinical prediction using similarity-based cohorts: A systematic review, 2020--: pre-MD~(1)
\item
Organizer, Instructor, and Helper, The Carpentries \\
National Association of Multicultural Engineering Program Advocates (NAMEPA), 2022~Feb~17--18 \\
Introduction to R, 2022~Jan~31--Feb~1 \\
Introduction to R, 2021~Sep~28
\item
Co-supervisor, Mathematics in Medicine Training Program (UConn Health) \\
Topological Modeling of Personalized Outcome Prediction, 2019--2022: PhD~(2), MD--PhD~(1)
\item
Mentor, High School Research Apprentice Program \\
Robustness analysis of the Mapper construction, 2019~Summer: HS~(2) \\
A tidyverse extension for ordination and biplot analysis, 2018~Summer: HS~(1) \\
Formal concept analysis of chronic comorbidities, 2016~Summer: HS~(1)
\item
Mentor, Research Experience for Undergraduates (REU) on Modeling and Simulation in Systems Biology \\
Modeling Incidence and Severity of Disease using Healthcare Data, 2017~Summer: Undergrad~(2) \\
Network analysis of mathematics research collaborations, 2010~Summer: Undergrad~(4)
\item
Adjunct instructor, Radford University \\
Math and Human Society, 2014~Spring
\item
Instructor, Virginia Tech \\
Multivariable Calculus, 2010~Spring \\
Calculus, 2008~Autumn--2009~Spring \\
Methods in Mathematical Modeling (designer and organizer), 2007~Autumn, 2008~Autumn \\
Vector Geometry (recitation), 2006~Autumn--2008~Spring
%As a recitation instructor I guided class discussions and exercises. As an instructor I was responsible for all aspects of the class except a portion of the final exam.
%As a recitation instructor my job was to help students hone their understanding of concepts they'd been exposed to in lecture; class time largely consisted of group work and student presentations of it, while I kept watch for questions to address or correct explanations to confirm. As a full instructor I prepared lesson plans, assignments, and exams; lectured, took questions, and gave exercises in class; graded; and held office hours. I tried always to have some challenges ready for students who wished to take them without letting this supplemental work give them an unfair advantage. In class, I became adept at building my responses out of the correct nuggets within every question or statement from a student. Gradually I've come to appreciate group exercises more than individual and to prefer interesting or counterintuitive problems to word problems that concoct an ``application'' rarely of use to understanding the mathematics itself. I've grown to want my students to shed their shame over being wrong, and to be more argumentative than comfortable, while still learning to be methodical.
%With a faculty co-organizer (Lizette Zietsman), I arranged a seminar series in math modeling featuring faculty in several disciplines. The course has continued, and students have participated in the Mathematical Contest in Modeling each following February.
%\item
%Tutor and supervisor, Math Emporium, 2000--2006
%I worked with groups of students on notes, exercises, and graded work; or with individuals for a few minutes at a time.
%Tutoring took two forms: In the tutoring lab, an environment for group work, we sat with one to four students enrolled in a common course to discuss their notes, homework, or returned graded work in detail; I've always hesitated to provide answers and during this time learned to ask more effective leading questions and encapsulate concepts so that they may ``click'' for students. On the main floor, a more individual work environment, we were expected to address specific questions quickly and encouragingly; within five minutes or so we learned where a student was getting caught up, guided them with a separate example or with leading questions, and made sure they had a next step in mind before stepping away. (We also had to indicate to some students the difference between a specific question and ``How do you do this?''; sometimes the time I'd spend would be helping them pose a specific question, and often that was enough for them to push forward on their own.
\end{itemize}

\vspace{.25cm}
{\sc Training}
\begin{itemize}[label=$\circ$,nolistsep]
\item
K College, University of Florida (UF) Clinical and Translational Science Institute, 2020--
\item
Instructor Training, The Carpentries, 2021~Jun~23--24
\item
Effective Business Writing Techniques, Instructional Solutions, 2021~Apr--May
\item
AMIA 10x10 1097: Introduction to Biomedical Informatics, 2020~Nov--2021~Mar
\item
Good Clinical Practices Course, National Institute of Allergies and Infectious Diseases, 2020~Nov~8
\item
Analysis of Big Healthcare Databases, ASA Connecticut Chapter Travel Course, 2019~Oct~16
\item
Responsible Conduct in Research (UConn Health MEDS 5310), 2019~Spring
\item
Craniofacial and Oral Biology (UConn Health MEDS 5415), 2018~Autumn
\item
The Science of Teaching -- A Course on Effective Teaching Practice (The Jackson Laboratory for Genomic Medicine), 2018~Autumn
\item
Data Carpentry Workshop (The Jackson Laboratory for Genomic Medicine), 2016
\item
Communicating Science (GRAD 5144), 2013~Spring
\item
Summer School and Conference in Geometric Representation Theory and Extended Affine Lie Algebras, U Ottawa, 2009~Jun~15--Jul~3
\item
Workshop on Representation Theory, Geometry and Combinatorics, UC Berkeley, 2008~Jun~2--6
\item
Topics in Combinatorial Representation Theory, MSRI, 2008~Mar~17--21
\end{itemize}
%
\vspace{.25cm}
{\sc Programming}
\begin{itemize}[label=$\circ$,nolistsep]
\item R (advanced; tidyverse); Python (intermediate); C++ (basic)
\item PostgreSQL (basic); MySQL (basic)
\item Macaulay 2 (intermediate); Mathematica (basic)
\item Git (intermediate); GitHub, Bitbucket
\end{itemize}

\vspace{.25cm}
{\sc Memberships}
\begin{itemize}[label=$\circ$,nolistsep]
\item
American Mathematical Society (AMS), 2019--
\item
American Medical Informatics Association (AMIA), 2018--
\item
UConn Health--JAX-GM Postdoctoral Association (UJPDA), 2015--2020
\item
Society for Industrial and Applied Mathematics (SIAM), 2014--
\end{itemize}

\vspace{.25cm}
{\sc Service}
\begin{itemize}[label=$\circ$,nolistsep]
\item
Co-organizer, Systems Medicine Student Research Symposium, 2022~May~10
\item
Board Member, UF Carpentries Club, 2021--2022
\item
Member, Faculty Council Research Task Force, 2021--
\item
Substitute Representative, UF Faculty Council, 2021
\item
Host, ``New Books in Mathematics'', New Books Network, 2019-- (21)
\item
Editing, proofreading, \& formatting, {\itshape The Ethical Challenges of the Stem Cell Revolution}, 2019--2020
\item
Poster \& digital presentation judge, Medical and Dental Student Research Day, 2017--2020 (4)
\item
Co-founder and President, UConn Health--JAX Genomic Medicine Postdoctoral Association, 2015--2020
\item
Matching Coordinator, UConn Health Speed Networking, 2019~Apr~25
\item
GitHub Coordinator, Medication Reconciliation Hackathon, Office of Health Strategy, 2019~Apr~6
\item
Organizer, Open Access and Science presentation, 2019~Jan~7
\item
Front Desk and Clinical Support, Hartford Gay and Lesbian Health Collective, 2015--2018
\item
Organizer, Scientific Writing \& Editing Support Group, 2018~Autumn
\item
Organizer, Drop-in R Consulting, 2018
\item
Co-organizer, UConn Health Speed Networking, 2018~Apr~19
\item
Postdoc Representative and Negotiating Team Member, University Health Professionals AFT Local 3837, 2015--2017
\item
Drop-in editing, Tool Kit for Scientific Communication course, 2017
\item
Organizing Committee member, Postdoc Research Day, 2017
\item
Co-organizer, ACSB 2015: A Conference on Algebraic and Combinatorial Approaches in Systems Biology, 2015~May~22--24
\item
Co-organizer, Virginia Tech Grad Student Speed Dating, 2012--2014
\item
Co-organizer, Graduate Student Combinatorics Seminar, 2010--2012
\item
Webmaster; Treasurer, SIAM Student Chapter, 2005--2007
\item
Graduate Representative, Math Club, 2004--2007
\item
Reviewer
\begin{itemize}[label=$\circ$,nolistsep]
\item
{\itshape Journal of Theoretical Biology}, 2022-- (1)
\item
{\itshape International Journal of Health Policy and Management}, 2020-- (1)
\item
{\itshape Journal of the American Medical Informatics Association}, 2019-- (2)
\item
{\itshape Bulletin of Mathematical Biology}, 2019-- (1)
\item
{\itshape Journal of Open Source Software}, 2018-- (8)
\item
{\itshape PLoS ONE}, 2017-- (1)
\item
AMIA Annual Symposium, 2016-- (4--6/yr)
\end{itemize}
\end{itemize}

\vspace{.25cm}
{\sc Resources}
\begin{itemize}[label=$\circ$,nolistsep]
\item
From work by Paul Magwene, {\tt\small latex-nihbiosketch}: A \LaTeX\ class implementing the new NIH Biographical Sketch Format {\sffamily Beamer} theme, 2020--2022
\item
{\tt\small NIH-proposal-template}: A Markdown--Pandoc template for NIH grant proposals, 2018--2022
\item
{\tt\small beamerthemeufl}: UF \LaTeX\ {\sffamily Beamer} theme, 2020--2021
\item
{\tt\small beamerthemeuconn}: UConn \LaTeX\ {\sffamily Beamer} theme, 2017--2020
\end{itemize}

\vspace{.25cm}
{\sc Presentations}
\begin{itemize}[label=$\circ$,nolistsep]
%\item
%``Adaptive Landmark Selection from Vectorized Patient Data of Varying Density'', AMIA Annual Symposium, Washington DC, 2022~Nov~5--9 (submitted)
%\item
%``Domain-Informed and -Agnostic Patient Similarity Measures for Individualized Mortality Prediction'', AMIA Annual Symposium, Washington DC, 2022~Nov~5--9 (submitted)
\item
``Toward tidy principles for matrix-decomposed data'', Special Session on New Developments in Graphic Multivariate Data, Joint Statistical Meetings, Washington  DC, 2022~Aug~6--11 (invited)
\item
``Toward a tidy package ecosystem for topological data analysis'', Special Session on Expanding {\sffamily tidyverse}, useR!, All-virtual, 2022~Jun~20--23
\item
``Post--Lung Transplant Disparities Amongst Sarcoidosis Patients'', Mini-Symposium on Advances in Pre- and Post-Lung Transplant Care: From Selection to CLAD, American Thoracic Society International Conference, 2022~May~13--18
\item
``Network Novelty in Biomedical Research: 3 Cases'', Southeast Center for Mathematics and Biology 4th Annual Symposium, remote, 2021~Dec~13--16 (poster)
\item
``Spatial graph analysis of glomerular capillaries'', Special Session on Algebra, Combinatorics, and Topology in Biological Structures, AMS Fall Southeastern Sectional Meeting, University of South Alabama (moved online), 2021~Nov~20--21 (invited)
\item
``Measuring Patient Similarity and Individualizing Predictive Models'', Weekly Research Conference, Division of Pulmonary, Critical Care, and Sleep Medicine Weekly Research Conference, UF, 2021~Jan~20
\item
``Network Analyses of Murine Glomeruli'', Department of Mathematics Biomathematics Seminar, UF, 2020~Oct~1
\item
``Network analyses of murine glomeruli'', SIAM Conference on the Life Sciences, 2020~Jun~8--11 (canceled)
\item
``Network methods in biomedical research: 3 use cases'', Skeletal, Craniofacial, \& Oral Biology Training Program Symposium, 2019~Oct~31
\item
``Network methods in biomedical research: 3 use cases'', Postdoc Research Day 2019, UConn Health, 2019~Sep~17
\item
``Network Analyses of Glomerular Capillaries'', Biology and Medicine Through Mathematics (BAMM!) Conference, Virginia Commonwealth University, 2019~May~15--17
\item
``Network Analyses of Murine Glomeruli'', $\pi$ Day Research Roundtable, UConn, 2019~Mar~14
\item
``Interrogating network models of epidemiological comorbidity'', Skeletal, Craniofacial, \& Oral Biology Training Program Symposium, 2018~Sep~25
%\item
%``Network analysis to measure disparities in professional healthcare infrastructure'', MiM Journal Club
\item
``Pairwise versus multivariate constructions of co-occurrence networks'', SIAM Workshop on Network Science, 2018~Jul~12--13
%\item
%``Conventional versus topological data analysis for disease subtyping: cases of type-2 diabetes mellitus'', MiM Journal Club, 2018
%\item
%``Co-occurrence Networks from Correlation Matrices'', CICATS study group, 2018
\item
``Modeling Incidence and Severity of Disease using Administrative Healthcare Data``, Open Data Salon, Hartford Public Library, 2017~Oct~26
%\item
%``Applications of network analysis to routinely collected health care data'', CQM presentation, UConn Health, 2017
%\item
%``Power-law distributions in empirical data'', MiM Journal Club + CICATS study group, 2017
%\item
%``RPKM versus TPM for comparing multiple gene expression across multiple RNA-seq samples'', Laubenbacher Group projects meeting, 2017
%\item
%``Emerging network methods in healthcare informatics'', CICATS study group, 2016
%\item
%``Modulus on graphs as a generalization of standard graph-theoretic quantities'', MiM Journal Club, 2016
%\item
%``Scientometrics in Space and Time'', CQM Faculty/Staff seminar, 2016
\item
Tutorial on data analysis and visualization in R, Postdoctoral Seminar, UConn Health, 2015~May~26
\item
``Evolving Collaboration Patterns in Medical Research'', AMIA Annual Symposium, 2014~Nov~19
\item
``Triad census for two-mode networks'', SIAM Workshop on Network Science, 2014~Jul~7
\item
``Surveying the Diagnostic Landscape'', Mining Networks and Graphs: A Big Data Analytic Challenge, SIAM International Conference on Data Mining, 2014~Apr~26
%\item
%``Caution in interpreting graph-theoretic diagnostics'', SIAM Student Seminar, 2013~Apr
\item
Schubert calculus (lecture series), VBI, 2012~Autumn
%\item
%Applied Discrete Mathematics Group seminar:
%\begin{itemize}[nolistsep]
%\item
%``Asking better questions'', 2013~Apr
%\item
%``The naturality of (some) combinatorics'', 2012~Oct
%\item
%``Foundations for a Connectivity Theory for Simplicial Complexes'', 2012~Apr
%\item
%``Mapping Change In Large Networks'' (paper presentation), 2011~Oct
%\item
%``Evolutionary Events in a Mathematical Sciences Collaboration Network'', 2011~Oct
%\item
%``The mathematics coauthorship graph exhibits separation-based linking that is well-modeled by a power law'', 2011~Mar
%\end{itemize}
\item
``Evolution of the mathematics research collaboration network'': GSA Research Symposium, Virginia Tech, 2012~Mar~28 (poster)
\item
``Gr\"obner geometry of Schubert polynomials'': Algebraic Geometry seminar, 2011~Nov
\item
``An introduction to generating functions'': MSSB REU, 2010~Jul (lecture)
\item
``Evolution of a mathematics collaboration network'': Dynamics On Networks, SAMSI, 2011~Mar~21--23 (poster)
\item
``A geometric construction of \(k\)-Schur polynomials'', informal seminar, UC Davis, 2009~Jun~8
\item
``Equations of matrix affine Schubert varieties'': RTGC, UC Berkeley, 2008~Jun~2 (poster); GSCC, University of Kentucky, 2009~Mar~28
%\item
%``Schubert calculus'': Algebraic Geometry, 2009 (guest lecture)
%\item
%Lecture ``Big matrix representations'': Combinatorics Seminar, 2007
%\item
%Lecture series ``Rings on a Plane'': Graduate Algebra Seminar, 2006
\end{itemize}

\vspace{.25cm}
{\sc Professional Activities}
\begin{itemize}[label=$\circ$,nolistsep]
\item
Research Conference Preview Journal Club, Pulmonary Division, 2021--
\item
Workshop on Topological Data Analysis, Institute for Mathematical and Statistical Innovation (IMSI), 2021~Apr~26--30
\item
Mentoring Reimagined -- International Mentoring Association 2021 Spring Symposium, 2021~Feb~23
\item
Faculty Boot Camp, UF Office of Faculty Affairs \& Professional Development, 2020--
\item
Division of Pulmonary, Critical Care, and Sleep Medicine weekly seminar, UF Department of Medicine, 2020--
\item
BioMathematics seminar, UF Department of Mathematics, 2020--
\item
Topological Data Analysis discussion group, UF Department of Mathematics, 2020--
\item
JAMIA Journal Club, 2019--
\item
Education Interest Group seminar, 2019--2020
\item
Connecticut Institute for Clinical and Translational Science (CICATS) Study Group, 2016--2020
\item
Mathematics in Medicine (MiM) Journal Club, 2016--2020
\item
Center for Cell Analysis and Modeling (CCAM) seminar series, 2016--2020
\item
{\sffamily rstudio::conf}, 2020~Jan~29--30
\item
American Association for the Advancement of Science Annual Meeting, 2019~Feb~14--17
\item
U.S. Department of Veterans Affairs Health Services Research \& Development Cyberseminars, 2018~Spring
\item
AMIA 2014 Annual Symposium, 2014~Nov~15--19
\item
SIAM Workshop on Network Science, 2014~Jul~6--7
\item
SDM 2014 Workshop on Mining Networks and Graphs, 2014~Apr
\item
Applied Discrete Mathematics Group Seminar, 2011--2013
\item
Workshop on Algebraic Methods in Evolutionary and Systems Biology, MBI, 2012~May~7--11
\item
Interdisciplinary Research Day, Virginia Tech, 19 Apr~2011
\item
Dynamics On Networks, SAMSI, 2011~Mar~21--23
\item
21st International Conference on Formal Power Series and Algebraic Combinatorics (FPSAC), RISC, 2009~Jul~20--24
\item
Graduate Student Combinatorics Conference, U Kentucky, 2009~Mar~27--29
\item
Combinatorial, Enumerative and Toric Geometry, MSRI, 2009~Mar~23--27
\item
FPSAC 20, Vi\~{n}a del Mar, Chile, 2008~Jun~23--27
\end{itemize}

\iffalse

%\vspace{.25cm}
%{\sc in press}
%\begin{enumerate}[label={[\arabic*]},labelindent=1cm,nolistsep]
%\setcounter{enumi}{\value{paper}}
%\setcounter{paper}{\value{enumi}}
%\end{enumerate}

\vspace{.25cm}
{\sc submitted}
\begin{enumerate}[label={[\arabic*]},labelindent=1cm,nolistsep]
\setcounter{enumi}{\value{paper}}
\item
Maximally assortative graphs and preferential attachment.\\ {\tt\small\nolinkurl{http://arxiv-web3.library.cornell.edu/abs/1308.4067}}
\setcounter{paper}{\value{enumi}}
\end{enumerate}

\vspace{.25cm}
{\sc in preparation}
\begin{enumerate}[label={[\arabic*]},labelindent=1cm,nolistsep]
\setcounter{enumi}{\value{paper}}
\item
--, E. Brown. More names for \((7,3,1)\).
\setcounter{paper}{\value{enumi}}
\end{enumerate}

%\pagebreak

%\vspace{.25cm}
%{\sc Activism}
%\begin{itemize}[label=$\circ$,nolistsep]
%\item Fact-checker, {\it The Interloper}, 2014~Spring
%\item Contributor, {\it Reflections} blog, 2014~Spring
%\item Volunteer Coordinator, ModernPoly ({\tt\small modernpoly.com}), 2010~Sep--
%\item Coordinator, Grad Student Speed Dating, 2012--2014~Feb
%\item ``Skepticism, Denial, and \sout{Flavor Aid} Fluoride'' (presentation), Freethinkers at Virginia Tech, 2012~Apr
%\item Co-founder and organizer, Blacksburg Polyamory, 2011--2014
%\item ``{\it P} Is for {\it Poly}'' (presentation), LGBTA at Virginia Tech, 2011~Apr
%\item Information Officer, Queer Grads and Allies, 2011--2012
%\item Co-founder and Information Officer, Sustainable Food Corps, 2009--2010
%\end{itemize}

\vspace{.25cm}
{\sc References}\\[1ex]% are available upon request.
\begin{tabularx}{\textwidth}{Xll}
Mark Shimozono, PhD advisor & Reinhard Laubenbacher, research supervisor \\
{\tt\small mshimo@vt.edu} & {\tt\small laubenbacher@uchc.edu} \\
460 McBryde Hall & Center for Quantitative Medicine \\
Virginia Tech & 195 Farmington Ave \\
Blacksburg, VA\ \ 24061 &  Farmington, CT\ \ 06030 \\[1ex]
Eileen Shugart, TA Coordinator \\
{\tt\small shugart@vt.edu} \\
460 McBryde Hall \\
Virginia Tech \\
Blacksburg, VA
\end{tabularx}

\fi

\end{document}
